\section{Proposal}

We propose a new attribute namespace introducer, to introduce an attribute
namespace in the current attribute specifier.

\begin{lstlisting}
// Current situation
void f() {
  [[rpr::kernel, rpr::target(cpu,gpu)]]
  do_task();
}

// Proposed change
void g() {
  [[using rpr: kernel, target(cpu,gpu)]]
  do_task();
}
\end{lstlisting}

\subsection{Effect of the using introducer}

The effect of a \cppid{using} introducer is to introduce all the attributes
names from a specific attribute namespace into the global attribute namespace.
Thus, after a \cppid{using} attribute, all the attributes from that namespace
can be used without explicit mention to the namespace.

\begin{lstlisting}
void g() {
  [[using rpr: kernel]] // equivalent to [[rpr::kernel]]
  do_task();
\end{lstlisting}

\subsection{Scope of the using introducer}
\label{scoping}

The effect of a \cppid{using} introducer is limited to the attribute list
where it appears.

\begin{lstlisting}
void f(X & x) {
  [[rpr::kernel, rpr::target(gpu), rpr::out(x)]] g1(x); // OK
  [[using rpr: kernel, target(gpu), out(x)]] g2(x); // OK
  [[using rpr: kernel]] [[target(gpu)]] g3(x); // Wrong. Target in different attr-list
}
\end{lstlisting}

\subsection{Simplified rules}

Instead of introducing complex lookup rules (as in~\cite{p0028r1}) we propose
a simplified set of rules:


Only a single attribute namespace introducer may be used within an attribute
specifier.

\begin{lstlisting}
[[using ns1: at1, at2, using ns2: at3]] // Ill-formed
\end{lstlisting}


All attributes in an attribute specifier containing a namespace introducer are
interpreted as if every attribute in that attribute specifier was prefixed by
the introduced namespace.

Thus the following specifier:

\begin{lstlisting}
[[using ns1: at1, at2, at3]]
\end{lstlisting}

is equivalent to:

\begin{lstlisting}
[[ns1::at1, ns1::at2, ns1::at3]]
\end{lstlisting}


