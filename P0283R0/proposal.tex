\section{Proposal}

This paper proposes possible solutions for the previously identified problems

\subsection{Handling unknown attribute namespaces}

Having scoped attributes as conditionally supported provides the degree of
freedom that allows an implementation not to support a specific attribute namespace.
However, this is not enough to clarify what are the valid options for an
implementation when it finds an attribute namespace it does not know about.

This paper proposes to add the following:

\vspace{1em}
\emph{All occurrences of an attribute namespace that is not conditionally supported
by the implementation will be ignored}.

\subsection{Non standard unqualified attributes}

The current wording of 7.6.1/5 makes all unqualified attributes that are not defined
in the standard as implementation defined. This allows implementations to add new
attributes to the global namespace.

As it has been mentioned above implementations do already make use of this clause
to define their own attributes in the global namespace. This paper proposes
to deprecate that use by changing the clause and adding additional rules.

This paper proposes to modify 7.6.1/5 as follows:

\vspace{1em}
\emph{Any occurrence of an attribute-token not specified in this International
Standard is deprecated}.
\vspace{1em}

This paper also proposes to add the following:

\vspace{1em}
\emph{For an attribute-scoped-token not specified in this International
Standard, the behavior is implementation-defined.}

