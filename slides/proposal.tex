\begin{frame}[t,fragile]{Proposal}
\begin{itemize}
  \item New attribute to introduce an attribute namespace.
    \begin{itemize}
      \item Alternatives on scoping later.
    \end{itemize}
\end{itemize}

\vfill
\begin{lstlisting}
// Current situation
void f() {
  [[rpr::kernel, rpr::target(cpu,gpu)]]
  do_task();
}

// Proposed change
void g() {
  [[using(rpr), kernel, target(cpu,gpu)]]
  do_task();
}
\end{lstlisting}
\end{frame}

\begin{frame}[t]{Some bikeshedding?}
\begin{itemize}
  \item \cppkey{[[using(something)]]}
  \item \cppkey{[[with(something)]]}
  \item \cppkey{[[namespace(something)]]}
  \item \cppkey{[[put\_your\_bikeshed\_here(something)]]}
  \item \ldots
\end{itemize}
\end{frame}

\begin{frame}[t,fragile]{Effects}
\begin{itemize}
  \item The effect of a \cppid{using} attribute is to introduce all the attributes
        names from a specific attribute namespace to the global attribute namespace.
    \begin{itemize}
      \item All the attributes from that namespace
            can used without explicit mention to the namespace.
    \end{itemize}
\end{itemize}

\begin{lstlisting}
void g() {
  [[using(rpr), kernel]] // equivalent to [[rpr::kernel]]
  do_task();

  // ...
}
\end{lstlisting}
\end{frame}

\begin{frame}[t,fragile]{Optional generalization}
\begin{itemize}
  \item It could be possible to eventually introduce multiple attribute
        namespaces in a single attribute specifier.
\end{itemize}
\vfill
\begin{lstlisting}
void g() {
  [[using(rpr,optimization), kernel, unroll(4)]]
  for (int i=0; i<max; ++i) {
    do_task(i);
  }
}
\end{lstlisting}
\vfill
\begin{itemize}
  \item \alert{I am not proposing this!}
\end{itemize}
\end{frame}

\begin{frame}[t,shrink]{Lookup}
\begin{itemize}
  \item Need rules to decide when an attribute specifier is seen to resolve the namespace
        it belongs to.
\end{itemize}
\vfill\pause
\begin{itemize}
  \item If an attribute specifier contains a non-scoped attribute name.
    \begin{enumerate}[1]
      \item Check against list of standard attributes.
      \item Check against valid attributes in attribute namespace brought to scope.
    \end{enumerate}
  \pause
  \item Outcome:
    \begin{enumerate}[a)]
      \item The lookup produces exactly one match. The program is considered
      to be well formed.
      
      \item The lookup produces more than one match in different attribute namespaces.
      The effect, in this this case, is implementation defined
      
      \item The lookup produces zero matches. The attribute is ignored and the
      program is well-formed, although a diagnostic may be optionally emitted.
        \begin{itemize}
          \item Alternatively, implementation defined.
        \end{itemize}
    \end{enumerate}
\end{itemize}
\end{frame}

\begin{frame}[t]{Scope of the using attribute}
\begin{itemize}
  \item Multiple design alternatives:
    \begin{enumerate}[1.]
      \item Effect restricted to the current attribute specifier. \alert{Preferred}
      \item Effect until the end of the current translation unit.
      \item Effect restricted to the current scope.
    \end{enumerate}
\end{itemize}
\end{frame}

\begin{frame}[t,fragile]{Sequence specifier scope}
\begin{lstlisting}
void f(X & x) {
  [[rpr::kernel, rpr::target(gpu), rpr::out(x)]] g1(x); // OK

  [[using(rpr), kernel, target(gpu), out(x)]] g2(x); // OK

  // Wrong. Target in different attr-specifier
  [[using(rpr), kernel]] [[target(gpu)]] g3(x); 
}
\end{lstlisting}

\begin{itemize}
  \item Option is simple to implement.
\end{itemize}
\end{frame}

\begin{frame}[t,fragile]{Original example}
\begin{lstlisting}[basicstyle=\footnotesize]
void f() {
  [[using(rpr), pipeline(bound, 8, blocking), stream(A,B)]]
  for (int i=0; i<iterations; ++i) {
    [[using(rpr), kernel, out(a), target(cpu) ]]
    a = get_value();
    
    [[using(rpr), farm(4,ordered), in(A,C), out(A,B), target(cpu,gpu)]]
    for (int j=0;j<max;++j) {
      b = f(a,c);
    }

    [[using(rpr), kernel, in(A,B)]]
    g(a,b);
  }
}
\end{lstlisting}
\end{frame}
