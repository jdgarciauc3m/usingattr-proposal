\begin{frame}[t,fragile]{Proposal}
\begin{itemize}
  \item New attribute to introduce an attribute namespace.
    \begin{itemize}
      \item Alternatives on scoping later.
    \end{itemize}
\end{itemize}

\vfill
\begin{lstlisting}
// Current situation
void f() {
  [[rpr::kernel, rpr::target(cpu,gpu)]]
  do_task();
}

// Proposed change
void g() {
  [[using rpr: kernel, target(cpu,gpu)]]
  do_task();
}
\end{lstlisting}
\end{frame}

\begin{frame}[t]{Some bikeshedding?}
\begin{itemize}
  \item \cppkey{[[namespace something:]]}
  \item \cppkey{[[with something:]]}
  \item \cppkey{[[use something:]]}
  \item \cppkey{[[put\_your\_bikeshed\_here something:]]}
  \item \ldots
\end{itemize}
\end{frame}

\begin{frame}[t,fragile]{Effects}
\begin{itemize}
  \item The effect of a \cppid{using} introducer is to introduce all the attributes
        names from a specific attribute namespace to the global attribute namespace.
    \begin{itemize}
      \item All the attributes from that namespace
            can used without explicit mention to the namespace.
    \end{itemize}
\end{itemize}

\begin{lstlisting}
void g() {
  [[using rpr: kernel, target(cpu)]] // equivalent to [[rpr::kernel, rpr::target(cpu)]]
  do_task();

  // ...
}
\end{lstlisting}
\end{frame}

\begin{frame}[t,shrink]{Lookup}
\begin{itemize}
  \item Need rules to decide when an attribute specifier is seen to resolve the namespace
        it belongs to.
\end{itemize}
\vfill\pause
\begin{itemize}
  \item If an attribute specifier contains a non-scoped attribute name.
    \begin{enumerate}[1]
      \item Check against list of standard attributes.
      \item Check against valid attributes in attribute namespace brought to scope through \cppkey{using} introducer.
    \end{enumerate}
  \pause
  \item Outcome:
    \begin{enumerate}[a)]
      \item Exactly one match $\Rightarrow$ Program is well formed.
      
      \item More than one match in different attribute namespaces
      $\Rightarrow$
      Effect is implementation defined. [some namespace could be not supported].
      
      \item Zero matches 
      $\Rightarrow$
      The attribute is ignored and the
      program is well-formed, although a diagnostic may be optionally emitted.
        \begin{itemize}
          \item Alternatively, implementation defined.
        \end{itemize}
    \end{enumerate}
\end{itemize}
\end{frame}

\begin{frame}[t,fragile]{Simplification}
\begin{itemize}
  \item No lookup at all.
  \item Only a single introducer in an attribute specifier.
  \item All attributes in the attribute specifier as-if the were prefixed by introduced namespace.
  \item Example:
\begin{lstlisting}
[[using ns1: at1, at2, at3]]
\end{lstlisting}
  \item Is equivalent to:
\begin{lstlisting}
[[ns1::at1, ns1::at2, ns1::at3]]
\end{lstlisting}
\end{itemize}
\end{frame}

\begin{frame}[t,fragile]{Scope of introducers}
\begin{lstlisting}
void f(X & x) {
  [[rpr::kernel, rpr::target(gpu), rpr::out(x)]] g1(x); // OK

  [[using rpr: kernel, target(gpu), out(x)]] g2(x); // OK

  // Wrong. Target in different attribute list
  [[using rpr: kernel]] [[target(gpu)]] g3(x); 
}
\end{lstlisting}

\end{frame}

\begin{frame}[t,fragile]{Original example}
\begin{lstlisting}[basicstyle=\footnotesize]
void f() {
  [[using rpr: pipeline(bound, 8, blocking), stream(A,B)]]
  for (int i=0; i<iterations; ++i) {
    [[using rpr:, kernel, out(a), target(cpu) ]]
    a = get_value();
    
    [[using rpr: farm(4,ordered), in(A,C), out(A,B), target(cpu,gpu)]]
    for (int j=0;j<max;++j) {
      b = f(a,c);
    }

    [[using rpr: kernel, in(A,B)]]
    g(a,b);
  }
}
\end{lstlisting}
\end{frame}
