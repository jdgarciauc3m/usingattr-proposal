\section{Proposal}

We propose a new attribute namespace introducer, to introduce an attribute
namespace in the current attribute specifier.

\begin{lstlisting}
// Current situation
void f() {
  [[rpr::kernel, rpr::target(cpu,gpu)]]
  do_task();
}

// Proposed change
void g() {
  [[using rpr: kernel, target(cpu,gpu)]]
  do_task();
}
\end{lstlisting}

In this paper we have named the proposed introducer \cppkey{using} for the similarity to using directives. However,
this choice is for exposition only and a different name could be used.

\subsection{Effect of the using introducer}

The effect of a \cppid{using} introducer is to introduce all the attributes
names from a specific attribute namespace into the global attribute namespace.
Thus, after a \cppid{using} attribute, all the attributes from that namespace
can used without explicit mention to the namespace.

\begin{lstlisting}
void g() {
  [[using rpr: kernel]] // equivalent to [[rpr::kernel]]
  do_task();
\end{lstlisting}

\subsection{Name lookup}

The introduction of the \cppkey{using} introducer requires additional
rules when an attribute specifier is seen. 

If an attribute specifier contains a non-scoped attribute name, it is
first checked against the list of standard attributes. If it is found,
the attribute lookup is finished.

If a non-scoped attribute is a non-standard attribute, it is checked
against the attribute name lists of all the attribute namespaces that have been
introduced through an \cppkey{using} introducer. The outcome of this process
could be one of the following:

\begin{enumerate}

\item The lookup produces exactly one match. The program is considered
to be well formed.

\item The lookup produces more than one match in different attribute namespaces.
The effect, in this this case, is implementation defined

\item The lookup produces zero matches. The attribute is ignored and the
program is well-formed, although a diagnostic may be optionally issued.

\end{enumerate}

\subsection{Scope of the using introducer}
\label{scoping}

The effect of a \cppid{using} introducer is limited to the attribute list
where it appears.

\begin{lstlisting}
void f(X & x) {
  [[rpr::kernel, rpr::target(gpu), rpr::out(x)]] g1(x); // OK
  [[using rpr: kernel, target(gpu), out(x)]] g2(x); // OK
  [[using rpr: kernel]] [[target(gpu)]] g3(x); // Wrong. Target in different attr-list
}
\end{lstlisting}
