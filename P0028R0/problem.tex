\section{Problem}

Attributes have proved to be a very useful way to perform
source code annotations. One example of this is the set
of attributes~\cite{repara:d33} defined in the context of the \emph{REPARA}
project (\url{http://www.repara-project.eu}).

A simple example of such use is the annotation of computational
kernels that can be later transformed to different
programming models.

\begin{lstlisting}
void f() {
  [[rpr::kernel]]
  for (int i=0; i<iterations; ++i) {
    do_something();
  }
}
\end{lstlisting}

However, in complex cases multiple attributes need to be
used in a single annotation. This results in a verbosity
that will make most implementations to look for very short
attribute namespaces names.

\begin{lstlisting}
void f() {
  [[rpr::pipeline(bound, 8, blocking), rpr::stream(A,B)]]
  for (int i=0; i<iterations; ++i) {
    [[rpr::kernel, rpr::out(a), rpr::target(cpu) ]]
    a = get_value();
    
    [[rpr::kernel, rpr::farm(4,ordered), rpr::in(A,C), rpr::out(A,B), rpr::target(cpu,gpu)]]
    for (int j=0;j<max;++j) {
      b = f(a,c);
    }

    [[rpr::kernel, rpr::in(A,B)]]
    g(a,b);
  }
}
\end{lstlisting}

An alternate solution could be to combine multiple attributes with a more
complex syntax, but this would introduce complexities in the attribute
syntax itself while making worse the ability to understand the annotations.
